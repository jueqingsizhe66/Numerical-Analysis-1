\documentclass{article}
\begin{document}
Daniel Purcell

MATH 4670

November 22, 2005

\section*{1.	Problem Statement}
Find the value of 
$$
\int_{-1}^{3} e^x \sin^2{x} \, dx
$$
using a) the composite trapezoidal rule and b) the composite Simpson's rule to
within $10^{-10}$.  Which method required more evaluations of $e^x \sin^2{x}$?


Note: According to my calculator, the "exact" value is 8.97050567528
\subsection*{1.a.	Composite Trapezoidal Rule}
$$
\int_{-1}^{3} e^x \sin^2{x} \, dx \approx 
	\frac{h}{2} \left[ f(-1) + f(3) + 2 \sum_{j=1}^{n-1} f(x_j) \right],
$$

\subsubsection*{C Code}

\begin{verbatim}

#include<stdlib.h>
#include<math.h>

double f(double x);
void trapezoid(double a, double b);
void output(int n, double val, double exact);

int main()
{
   double a, b;
   a = -1;
   b = 3;
   trapezoid(a, b); 
   return 0;
}

double f(double x)
{
   return exp(x) * pow(sin(x), 2);
}

void trapezoid(double a, double b)
{
   int n = 1000000;
   double sum, h, exact;
   sum = 0.0;
   int i;
   exact = 8.97050567528;
   while (exact - sum > 0.0000000001) 
   {
      h = (b - a) / (n + 2);
      sum = 0.0;
      for (i = 1; i < n - 1; i++)
         sum += f(a + (i + 1) * h);
      sum = (h/2) * ( f(a) + f(b) + 2 * sum );
      output(n, sum, exact);
      n += 1000000;
      
   }
}

void output(int n, double val, double exact)
{
   printf("%d \t %3.11f \t %3.11f \n", n, val, exact - val);
}

\end{verbatim}

\subsubsection*{Results}

\begin{verbatim}
   n             approximation       error
----------------------------------------------
10000000         8.97050525109   0.00000042419 
20000000         8.97050546318   0.00000021210 
30000000         8.97050553388   0.00000014140 
40000000         8.97050556923   0.00000010605 
50000000         8.97050559044   0.00000008484 
60000000         8.97050560458   0.00000007070 
70000000         8.97050561468   0.00000006060 
80000000         8.97050562226   0.00000005302 
90000000         8.97050562815   0.00000004713 
100000000        8.97050563286   0.00000004242 

\end{verbatim}

Obviously, the use of the composite trapezoidal rule is rather painful for this
problem, as it still did not produce results accurate to within $10^{-10}$ even
after 100,000,000 iterations.  It appears to eventually converge, but I do not
have a spare decade or two to confirm that.

\subsection*{1.b.  Composite Simpson's Rule}
$$
\int_{-1}^{3} e^x \sin^2{x} \, dx \approx
	\frac{h}{3} \left[ f(-1) + 2 \sum_{j=1}^{(n/2)-1} f(x_{2j})
                           + 4 \sum_{j=1}^{n/2} f(x_{2j - 1}) + f(3) \right],
$$
\subsubsection*{C Code}

\begin{verbatim}
#include<stdlib.h>
#include<math.h>

double f(double x);
void simpson(double a, double b);
void output(int n, double val, double exact);
double max(double x, double y);
double min(double x, double y);

int main()
{
   double a, b;
   a = -1;
   b = 3;
   simpson(a, b); 
   return 0;
}

double f(double x)
{
   return exp(x) * pow(sin(x), 2);
}

double max(double x, double y)
{
   return (x >= y) ? x : y;
}

double min(double x, double y)
{
   return (x <= y) ? x : y;
}

void simpson(double a, double b)
{
   int n = 10;
   double sum, sum_one, sum_two, h, exact;
   sum = 0.0;
   int i;
   exact = 8.97050567528;
   while (!(max(exact, sum) - min(exact, sum) < 0.00000000009 
		      && max(exact, sum) - min(exact, sum) >= 0)) 
   {
      h = (b - a) / n;
      sum = sum_one = sum_two = 0.0;
      for (i = 1; i <= ( (n/2) - 1) ; i++)
         sum_one += f(a + (2 * i) * h);
      sum_one *= 2;
      for (i = 1; i <= (n/2); i++)
         sum_two += f(a + (2 * i - 1) * h);
      sum_two *= 4; 
      sum = (h/3) * (f(a) + sum_one + sum_two + f(b));
      output(n, sum, exact);
      n += 10;
   }
}

void output(int n, double val, double exact)
{
   printf("%d \t %3.11f \t %3.11f \n", n, val, exact - val);
}
\end{verbatim}

\subsubsection*{Results}
\begin{verbatim}
  n      approximation        error
---------------------------------------
1080     8.97050567541   -0.00000000013 
1090     8.97050567541   -0.00000000013 
1100     8.97050567540   -0.00000000012 
1110     8.97050567540   -0.00000000012 
1120     8.97050567539   -0.00000000011 
1130     8.97050567539   -0.00000000011 
1140     8.97050567539   -0.00000000011 
1150     8.97050567538   -0.00000000010 
1160     8.97050567538   -0.00000000010 
1170     8.97050567538   -0.00000000010 
1180     8.97050567537   -0.00000000009 
1190     8.97050567537   -0.00000000009 
1200     8.97050567537   -0.00000000009 
\end{verbatim}
The composite Simpson's rule only takes around 1180 or so iterations, thus
it is more efficient to use Simpson's rule to solve this problem.



\section*{2.  Problem Statement}
Redo the integral from Problem One using Gaussian quadrature with $n = 1$, 2,
3, 4, 5.

\subsection*{Change of Variable}
$$
t = \frac{1}{2} (x + 1) - 1
$$
$$
x = 2t + 1
$$
$$
dx = 2 \, dt
$$

$$
\int_{-1}^{3} e^x \sin^2{x} \, dx = 
   \int_{-1}^{1} e^{2t + 1} \sin^2{(2t + 1)} \, 2\, dt
$$
\subsubsection*{C Code}

\begin{verbatim}
#include <stdlib.h>
#include <math.h>

double gauss(int n);
double f(double x);

int main()
{  
   int n;
   for (n = 2; n <= 5; n++)
       printf("%d \t %3.11f \n", n, gauss(n));
   return 0;
}

double gauss(int n)
{
   if (n == 2)
   {
     return f(0.5773502692) + f(-0.5773502692);
   }
   else if (n == 3)
   {
     return 0.5555555556 * f(0.7745966692) + 0.8888888889 * f(0.0000000000) + 
            0.5555555556 * f(-0.7745966692);
   }
   else if (n == 4)
   {
     return 0.3478548451 * f(0.8611363116) + 0.6521451549 * f(0.3399810436) + 
            0.6521451549 * f(-0.3399810436) + 0.3478548451 * f(-0.8611363116);
   }
   else if (n == 5)
   {
     return 0.2369268850 * f(0.9061798459) + 0.4786286705 * f(0.5384693101) + 
            0.5688888889 * f(0.0000000000) + 0.4786286705 * f(-0.5384693101) + 
            0.2369268850 * f(-0.9061798459);
   }
   else
     return 0.0;
}

double f(double x)
{
   return 2 * exp(2 * x + 1) * pow(sin(2 * x + 1), 2);
}

\end{verbatim}

\subsubsection*{Results}

\begin{verbatim}
n          approximation
------------------------
2         12.04855981792
3          8.02936122899
4          8.99497551338
5          8.97416372329
\end{verbatim}

Gaussian quadrature appears to reach accurate results faster than the previous
two methods.  However, while the computer has to do less work, the programmer
must do much more as $n$ increases.



\section*{3.   Problem Statement}
Find the arc length along the curve $y = \frac{1}{x}$ from the point where 
$x = \frac{1}{4}$ to the point where $x = 7$.

\subsection*{Arc Length Equation}
$$
s = \int_{a}^{b} \sqrt{1 + (f'(x))^2} \, dx
$$
$$
s = \int_{1/4}^{7} \sqrt{1 + (-x^2)^2} \, dx
$$

\subsection*{C Code}
I have decided to use a modification of the Simpson's rule program that I used
for Problem 1, Part b.

\begin{verbatim}
#include<stdlib.h>
#include<math.h>

double f(double x);
void simpson(double a, double b);
void output(int n, double val);

int main()
{
   double a, b;
   a = (1.0/4.0);
   b = 7;
   simpson(a, b); 
   return 0;
}

double f(double x)
{
   return sqrt(1 + pow(-pow(x, 2), 2));
}

void simpson(double a, double b)
{
   int n = 10;
   double sum, sum_one, sum_two, h, previous;
   sum = 0.0;
   previous = -1.0;
   int i;
   while (sum != previous) 
   {
      previous = sum;   
      h = (b - a) / n;
      sum = sum_one = sum_two = 0.0;
      for (i = 1; i <= ( (n/2) - 1) ; i++)
         sum_one += f(a + (2 * i) * h);
      sum_one *= 2;
      for (i = 1; i <= (n/2); i++)
         sum_two += f(a + (2 * i - 1) * h);
      sum_two *= 4; 
      sum = (h/3) * (f(a) + sum_one + sum_two + f(b));
      output(n, sum);
      n += 10;
   }
}

void output(int n, double val)
{
   printf("%d \t %3.11f \n", n, val);
}
\end{verbatim}

\subsection*{Results}

\begin{verbatim}
 n        approximation
------------------------
10       115.23572951875 
20       115.24764852813 
30       115.24781785209 
40       115.24784490226 
50       115.24785292640 
60       115.24785578154
 |             |
 |             |
1690     115.24785843073 
1700     115.24785843073 
1710     115.24785843073 
1720     115.24785843073 
1730     115.24785843073 
\end{verbatim}

The value converges to exactly 115.24785843073, according to my program. 
Using this integral in my calculator yields 115.247858431, which validates my
solution, since the calculator rounds off, because it handles two less digits 
than my program. 



\section*{4.a.   Problem Statement}
Find the exact value of
$$
\int_{1}^{4} \int_{0}^{3} \cos{x} + 2xy \, dy \, dx
$$

\subsection*{Solution}
$$
\int_{1}^{4} \int_{0}^{3} \cos{x} + 2xy \, dy \, dx =
$$
$$
\int_{1}^{4} \left. \left[ \cos{x}y + xy^2 \right] \right|_{0}^{3} \, dx = 
$$
$$
\int_{1}^{4} 3 \cos{x} + 9x \, dx = 
$$
$$
\left. \left[ 3 \sin{x} + \frac{9}{2} x^2 \right] \right|_{1}^{4} =
$$
$$
\left[ 3 \sin{4} + 72 \right] - \left[ 3 \sin{1} + \frac{9}{2} \right]
       \approx 62.7051795597
$$



\section*{4.b.   Problem Statement}
Find the value of the integral to within $10^{-5}$ by using the two dimensional version of the trapezoidal rule.  Use $n = m = 6$, $n = m = 12$, $n = m = 24$, 
and $n = m = 48$.

\subsection*{C Code}
\begin{verbatim}
#include <stdlib.h>
#include <math.h>

double f(double x, double y);
double trapezoid2d(int n, int m, double a, double b, double c, double d);


int main()
{
   int n, m;
   double a, b, c, d;
   a = 1;
   b = 4;
   c = 0;
   d = 3;
   for (m = n = 6; n <= 48; m = n = n * 2)	   
      printf("%d \t %d \t %3.15f \n", n, m, trapezoid2d(n, m, a, b, c, d));
   return 0;
}

double f(double x, double y)
{
   return cos(x) + 2 * x * y;
}

double trapezoid2d(int n, int m, double a, double b, double c, double d)
{
   double h, k, sum[3], i, j, xi, yj;
   h = (b - a) / n;
   k = (d - c) / m;
   sum[0] = sum[1] = sum[2] = 0.0;
   
   for (i = 1; i < n; i++)
   {
      xi = a + (double) i * h;
      sum[0] = sum[0] + f(xi, c) + f(xi, d);
   }
   
   for (j = 1; j < n; j++)
   {
      yj = c + (double) j * k;
      sum[1] = sum[1] + f(a, yj) + f(b, yj);
   }

   for (i = 1; i < n; i++)
   {
      for (j = 1; j < n; j++)
      {
         xi = a + (double) i * h;
         yj = c + (double) j * k;
         sum[2] = sum[2] + f(xi, yj);
      }
   }

   return (.25) * h * k * (2 * sum[0] + 2 * sum[1] + 4 * sum[2] + f(a, c)
          + f(a, d) + f(b, c) + f(b, d) );
}
\end{verbatim}

\subsection*{Results}

\begin{verbatim}
n        m          approximation
-----------------------------------
6        6       62.805490362276984 
12       12      62.730178635116282 
24       24      62.711424441887445 
48       48      62.706740475223107 
\end{verbatim}



\section*{5.   Problem Statement}
Using the answers from Problem 4.b., carry out the Romberg process.  Does it 
result in a greatly imrpoved approximation for the integral, or does it fail 
work?

\subsection*{C Code}
\begin{verbatim}
#include <stdlib.h>
#include <math.h>

int main()
{	
   int i, j;
   double romberg[4][4];
   romberg[0][0] = 62.805490362276984; 
   romberg[1][0] = 62.730178635116282;
   romberg[2][0] = 62.711424441887445; 
   romberg[3][0] = 62.706740475223107;

   for (j = 1; j < 4; j++)
       for (i = 1; i < 4; i++)
       {
          romberg[i][j] = (pow(4.0, j) * romberg[i][j-1] - 
                           romberg[i-1][j-1]) / (pow(4.0, j) - 1.0);
       }
   
   for (i = 0; i < 4; i++)
       for (j = 0; j <= 1; j++)
           printf("%d \t %d \t %3.12f \n", i, j, romberg[i][j]);
   
   return 0;
}
\end{verbatim}

\subsection*{Results}

\begin{verbatim}

i        j        approximation
--------------------------------
0        0       62.805490362277 
0        1       0.000000000000 
1        0       62.730178635116 
1        1       62.705074726063 
2        0       62.711424441887 
2        1       62.705173044144 
3        0       62.706740475223 
3        1       62.705179153002 
\end{verbatim}

The Romberg process appears to work correctly.




\section*{6.   Problem Statement}
For 
$$
V(a, b, c) = \int_{-1}^{1} \int_{-1}^{1}
                   \frac{1}{ \sqrt{(x - a)^2 + (y - b)^2 + c^2} } \, dy \, dx,
$$

Find V(5, 7, 3).

\subsection*{C Code}
I have decided to use a modification of the Gaussian quadrature program I used 
for Problem 2.

\begin{verbatim}
#include <stdlib.h>
#include <math.h>

double gauss(int n);
double f(double x);

int main()
{  
   int n;
   for (n = 2; n <= 5; n++)
      printf("%d \t %3.11f \n", n, gauss(n));
   return 0;
}

double gauss(int n)
{
   if (n == 2)
   {
     return f(0.5773502692) + f(-0.5773502692);
   }
   else if (n == 3)
   {
     return 0.5555555556 * f(0.7745966692) + 0.8888888889 * f(0.0000000000) + 
	    0.5555555556 * f(-0.7745966692);
   }
   else if (n == 4)
   {
     return 0.3478548451 * f(0.8611363116) + 0.6521451549 * f(0.3399810436) + 
	    0.6521451549 * f(-0.3399810436) + 0.3478548451 * f(-0.8611363116);
   }
   else if (n == 5)
   {
     return 0.2369268850 * f(0.9061798459) + 0.4786286705 * f(0.5384693101) +
	    0.5688888889 * f(0.0000000000) + 0.4786286705 * f(-0.5384693101) +
	    0.2369268850 * f(-0.9061798459);
   }
   else
     return 0.0;
}

double f(double x)
{
   return 1.0 / ( sqrt(pow(x - 5.0, 2) + pow(x - 7.0, 2) + pow(3.0, 2) ) );
}
\end{verbatim}

\subsection*{Results}

\begin{verbatim}
n          approximation
------------------------
2          0.22094713416
3          0.22095182309
4          0.22095182155
5          0.22095182140
\end{verbatim}

I assume that $V(5, 7, 3) \approx 0.22095182140$.





\end{document}

