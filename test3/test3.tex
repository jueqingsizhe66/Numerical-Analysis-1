\documentclass{article}
\begin{document}

Daniel Purcell

MATH 4670

December 7, 2005

\section*{5.2  Problem 1.a}
Use Euler's method to approximation solutions for
$$
y' = te^{3t} - 2y, \mbox{ for }0 \leq t \leq 1,	
		   \mbox{ with } y(0) = 0 \mbox{ and } h = 0.5
$$

\subsection*{C Code}
\begin{verbatim}
#include <stdio.h>
#include <math.h>


double f(double t, double w);

int main()
{
   int n = 30, i;
   double w = 0, a = 0, b = 1, t, h;
   h = (b - a) / (double) n;
   for (i = 0; i <= n; i++)
   {
      t = a + i * h;
      w = w + h * f(t, w);
      printf("%d \t %1.8f \t %3.8f \t \n", i, t, w);
   }
   
   return 0;
}

double f(double t, double w)
{
   return t * exp(3 * t) - 2 * w;
}
\end{verbatim}

\subsection*{Results}
\begin{verbatim}
i              t               w
-----------------------------------
0 	 0.00000000 	 0.00000000 	 
1 	 0.05000000 	 0.00290459 	 
2 	 0.10000000 	 0.00936342 	 
3 	 0.15000000 	 0.02018942 	 
4 	 0.20000000 	 0.03639167 	 
5 	 0.25000000 	 0.05921500 	 
6 	 0.30000000 	 0.09018755 	 
7 	 0.35000000 	 0.13117769 	 
8 	 0.40000000 	 0.18446226 	 
9 	 0.45000000 	 0.25280811 	 
10 	 0.50000000 	 0.33956952 	 
11 	 0.55000000 	 0.44880451 	 
12 	 0.60000000 	 0.58541349 	 
13 	 0.65000000 	 0.75530448 	 
14 	 0.70000000 	 0.96558998 	 
15 	 0.75000000 	 1.22482108 	 
16 	 0.80000000 	 1.54326603 	 
17 	 0.85000000 	 1.93324133 	 
18 	 0.90000000 	 2.40950513 	 
19 	 0.95000000 	 2.98972425 	 
20 	 1.00000000 	 3.69502867 	
\end{verbatim}



\section*{5.2	Problem 1.c}
Use Euler's method to approximate the solutions for
$$
y' = 1 + (y/t), \mbox{ for } 0 \leq t \leq 1, \mbox{ with } y(0) = 1
$$

\subsection*{C Code}
\begin{verbatim}
#include <stdio.h>
#include <math.h>


double f(double t, double w);

int main()
{
   int n = 20, i;
   double w = 2, a = 1, b = 2, t, h;
   h = (b - a) / (double) n;
   for (i = 0; i <= n; i++)
   {
      t = a + i * h;
      w = w + h * f(t, w);
      printf("%d \t %1.8f \t %3.8f \t \n", i, t, w);
   }
   
   return 0;
}

double f(double t, double w)
{
   return 1 + (w/t);
}
\end{verbatim}

\subsection*{Results}
\begin{verbatim}
i             t               w
-----------------------------------
0 	 1.00000000 	 2.15000000 	 
1 	 1.05000000 	 2.30238095 	 
2 	 1.10000000 	 2.45703463 	 
3 	 1.15000000 	 2.61386222 	 
4 	 1.20000000 	 2.77277315 	 
5 	 1.25000000 	 2.93368408 	 
6 	 1.30000000 	 3.09651808 	 
7 	 1.35000000 	 3.26120393 	 
8 	 1.40000000 	 3.42767550 	 
9 	 1.45000000 	 3.59587121 	 
10 	 1.50000000 	 3.76573358 	 
11 	 1.55000000 	 3.93720886 	 
12 	 1.60000000 	 4.11024664 	 
13 	 1.65000000 	 4.28479957 	 
14 	 1.70000000 	 4.46082308 	 
15 	 1.75000000 	 4.63827517 	 
16 	 1.80000000 	 4.81711615 	 
17 	 1.85000000 	 4.99730848 	 
18 	 1.90000000 	 5.17881659 	 
19 	 1.95000000 	 5.36160676 	 
20 	 2.00000000 	 5.54564693 	
\end{verbatim}



\section*{5.2 Problem 1.d}
Use Euler's method to approximate the solutions for
$$
y' = \cos{2t} + \sin{3t}, \mbox{ for } 0 \leq t \leq 1, \mbox{ with } y(0) = 1
$$

\subsection*{C Code}
\begin{verbatim}
#include <stdio.h>
#include <math.h>


double f(double t, double w);

int main()
{
   int n = 20, i;
   double w = 1, a = 0, b = 1, t, h;
   h = (b - a) / (double) n;
   for (i = 0; i <= n; i++)
   {
      t = a + i * h;
      w = w + h * f(t, w);
      printf("%d \t %1.8f \t %3.8f \t \n", i, t, w);
   }
   
   return 0;
}

double f(double t, double w)
{
   return cos(2*t) + sin(3*t);
}
\end{verbatim}

\subsection*{Results}
\begin{verbatim}
i             t               w
-----------------------------------
0 	 0.00000000 	 1.05000000 	 
1 	 0.05000000 	 1.10722211 	 
2 	 0.10000000 	 1.17100145 	 
3 	 0.15000000 	 1.24051656 	 
4 	 0.20000000 	 1.31480173 	 
5 	 0.25000000 	 1.39276279 	 
6 	 0.30000000 	 1.47319592 	 
7 	 0.35000000 	 1.55480919 	 
8 	 0.40000000 	 1.63624648 	 
9 	 0.45000000 	 1.71611315 	 
10 	 0.50000000 	 1.79300301 	 
11 	 0.55000000 	 1.86552607 	 
12 	 0.60000000 	 1.93233634 	 
13 	 0.65000000 	 1.99215927 	 
14 	 0.70000000 	 2.04381809 	 
15 	 0.75000000 	 2.08625861 	 
16 	 0.80000000 	 2.11857179 	 
17 	 0.85000000 	 2.14001376 	 
18 	 0.90000000 	 2.15002264 	 
19 	 0.95000000 	 2.14823207 	 
20 	 1.00000000 	 2.13448073 	
\end{verbatim}



\section*{5.2 Problem 8.e}
Use Taylor's method of order 4 with $h = 0.1$ to approximate 
$$
y' = \frac{2}{t}y + t^{2}e^{t},  1 \leq t \leq 2, y(1) = 0
$$
with the exact solution
$$
y(t) = t^{2}(e^{t} - e)
$$

\subsection*{C Code}
\begin{verbatim}
#include <stdio.h>
#include <math.h>


double f(double t, double w);
double y2(double t, double y);
double y3(double t, double y);
double exact(double t);
double taylor(double t, double y, double h);

int main()
{
   int n = 25, i;
   double w = 0, a = 1, b = 2, t, h;
   h = (b - a) / (double) n;
   for (i = 0; i <= n; i++)
   {
      t = a + i * h;
      w = w + h * taylor(t, w, h);
      printf("%d \t %1.8f \t %3.8f \t %3.8f \t %3.8f \n", i, t, w, exact(t), 
		      				               exact(t) - w);
   }
   
   return 0;
}

double f(double t, double w)
{
   return 2/t * w + t*t * exp(t);
}

double y2(double t, double y)
{
   return (-2/(t * t)) * y + 2 * t * exp(t) + t * t * exp(t) + 2 * t * f(t, y);
}

double y3(double t, double y)
{
  return (-4/(t*t*t) * y + 2*exp(t) + t*t*exp(t)) + (-2/(t*t) * f(t, y))
	  + f(t, y) * 2 + 2 * t * y2(t, y);
}

double exact(double t)
{
   return t * t * (exp(t) - exp(1));
}

double taylor(double t, double y, double h)
{
   return f(t, y) + h/2 * y2(t, y) + (h*h*h)/6 * y3(t, y);
}
\end{verbatim}

\subsection*{Results}
\begin{verbatim}
i             t               w             y(t)            error
--------------------------------------------------------------------
0 	 1.00000000 	 0.11961948 	 0.00000000 	 -0.11961948 
1 	 1.04000000 	 0.26369547 	 0.11998750 	 -0.14370798 
2 	 1.08000000 	 0.43488933 	 0.26407030 	 -0.17081902 
3 	 1.12000000 	 0.63605321 	 0.43474039 	 -0.20131282 
4 	 1.16000000 	 0.87024209 	 0.63465419 	 -0.23558791 
5 	 1.20000000 	 1.14072644 	 0.86664254 	 -0.27408390 
6 	 1.24000000 	 1.45100557 	 1.13372112 	 -0.31728445 
7 	 1.28000000 	 1.80482194 	 1.43910158 	 -0.36572036 
8 	 1.32000000 	 2.20617611 	 1.78620315 	 -0.41997296 
9 	 1.36000000 	 2.65934269 	 2.17866506 	 -0.48067762 
10 	 1.40000000 	 3.16888718 	 2.62035955 	 -0.54852763 
11 	 1.44000000 	 3.73968379 	 3.11540565 	 -0.62427814 
12 	 1.48000000 	 4.37693427 	 3.66818370 	 -0.70875057 
13 	 1.52000000 	 5.08618786 	 4.28335075 	 -0.80283710 
14 	 1.56000000 	 5.87336236 	 4.96585672 	 -0.90750563 
15 	 1.60000000 	 6.74476643 	 5.72096153 	 -1.02380491 
16 	 1.64000000 	 7.70712318 	 6.55425311 	 -1.15287007 
17 	 1.68000000 	 8.76759506 	 7.47166654 	 -1.29592852 
18 	 1.72000000 	 9.93381022 	 8.47950405 	 -1.45430617 
19 	 1.76000000 	 11.21389031 	 9.58445628 	 -1.62943403 
20 	 1.80000000 	 12.61647998 	 10.79362466 	 -1.82285532 
21 	 1.84000000 	 14.15077790 	 12.11454498 	 -2.03623292 
22 	 1.88000000 	 15.82656961 	 13.55521229 	 -2.27135732 
23 	 1.92000000 	 17.65426228 	 15.12410717 	 -2.53015511 
24 	 1.96000000 	 19.64492128 	 16.83022338 	 -2.81469790 
25 	 2.00000000 	 21.81030898 	 18.68309708 	 -3.12721190
\end{verbatim}

The results become less accurate as $t$ increases.

\section*{5.3 Problem 1.b}
Use the Midpoint method to approximate the solution for
$$
y' = te^{3t} - 2y,  0 \leq t \leq 1, y(0) = 0
$$
and compare with the actual solution
$$
y(t) = t + 1/(1 - t)
$$

\subsection*{C Code}
\begin{verbatim}
#include <stdio.h>
#include <math.h>


double f(double t, double w);
double exact(double t);

int main()
{
   int n = 20, i;
   double w = 1, a = 2, b = 3, t, h;
   h = (b - a) / (double) n;
   for (i = 0; i <= n; i++)
   {
      t = a + i * h;
      w = w + h * f(t + .5 * h, w + h * .5 * f(t, w));
      printf("%d \t %1.8f \t %3.8f \t %3.8f \t %3.8f \n", i, t, w, exact(t), 
                                                                  exact(t) - w);
   }
   
   return 0;
}

double f(double t, double w)
{
   return 1.0 + (t - w)*(t - w);
} 

double exact(double t)
{
  return t + 1.0/(1.0 - t);
}
\end{verbatim}

\subsection*{Results}
\begin{verbatim}
i            t              w               y(t)             error
--------------------------------------------------------------------
0 	 2.00000000 	 1.09753125 	 1.00000000 	 -0.09753125 
1 	 2.05000000 	 1.19075661 	 1.09761905 	 -0.09313757 
2 	 2.10000000 	 1.28023492 	 1.19090909 	 -0.08932583 
3 	 2.15000000 	 1.36643244 	 1.28043478 	 -0.08599766 
4 	 2.20000000 	 1.44974129 	 1.36666667 	 -0.08307463 
5 	 2.25000000 	 1.53049357 	 1.45000000 	 -0.08049357 
6 	 2.30000000 	 1.60897239 	 1.53076923 	 -0.07820316 
7 	 2.35000000 	 1.68542062 	 1.60925926 	 -0.07616137 
8 	 2.40000000 	 1.76004775 	 1.68571429 	 -0.07433347 
9 	 2.45000000 	 1.83303544 	 1.76034483 	 -0.07269061 
10 	 2.50000000 	 1.90454197 	 1.83333333 	 -0.07120864 
11 	 2.55000000 	 1.97470593 	 1.90483871 	 -0.06986722 
12 	 2.60000000 	 2.04364913 	 1.97500000 	 -0.06864913 
13 	 2.65000000 	 2.11147909 	 2.04393939 	 -0.06753970 
14 	 2.70000000 	 2.17829109 	 2.11176471 	 -0.06652638 
15 	 2.75000000 	 2.24416982 	 2.17857143 	 -0.06559839 
16 	 2.80000000 	 2.30919086 	 2.24444444 	 -0.06474641 
17 	 2.85000000 	 2.37342183 	 2.30945946 	 -0.06396237 
18 	 2.90000000 	 2.43692343 	 2.37368421 	 -0.06323922 
19 	 2.95000000 	 2.49975031 	 2.43717949 	 -0.06257082 
20 	 3.00000000 	 2.56195178 	 2.50000000 	 -0.06195178
\end{verbatim}

As $t$ increases, the results become more accurate.
\end{document}
